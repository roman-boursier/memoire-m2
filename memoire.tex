\documentclass[a4paper, 12pt]{book}
\usepackage{graphicx}
\usepackage[french]{babel}
\usepackage[utf8]{inputenc}
\usepackage[T1]{fontenc}
\usepackage{multirow}
\usepackage{listings}
\usepackage{float}
\usepackage{url}
\usepackage[french]{algorithm}
\usepackage{style/myalgorithm}
\usepackage{amsmath,amsfonts,amssymb}
\newcommand{\fBm}{\emph{fBm}~}
\newcommand{\etal}{\emph{et al.}~}
\newcommand{\glAd}{\emph{GL4D}~}
\newcommand{\apiopengl}{API OpenGL\textsuperscript{\textregistered}~}
\newcommand{\opengl}{OpenGL\textsuperscript{\textregistered}~}
\newcommand{\opengles}{OpenGL\textsuperscript{\textregistered}ES~}
\newcommand{\clang}{langage \texttt{C}}
\newcommand{\codesource}{\textsc{Code source}~}
\floatstyle{ruled}
\newfloat{programslist}{htbp}{locs}
\newcommand{\listofprograms}{\listof{programslist}{Liste des codes source}}
\newcounter{program}[subsection]
\renewcommand{\theprogram}{\arabic{chapter}.\arabic{program}}

\newenvironment{program}[1]{
  \if\relax\detokenize{#1}\relax
  \gdef\mycaption{\relax}
  \else
  \gdef\mycaption{#1}
  \fi
  \refstepcounter{program}
  \addcontentsline{locs}{section}{#1}
  \footnotesize
}{
  \begin{description}
    \item[\codesource \theprogram]--~\mycaption
  \end{description}
}

\begin{document}
\begin{titlepage}
  \begin{center}
    \begin{tabular*}{\textwidth}{l@{\extracolsep{\fill}}r}
      \includegraphics[height=1.5cm]{images/m2ise.png}
    \end{tabular*}
    \small 
    \rule{\textwidth}{.5pt}~\\
    \large 
    \textsc{Université Paris 8 - Vincennes à Saint-Denis}\vspace{0.5cm}\\
    \textbf{Master Informatique des Systèmes Embarqués}\vspace{3.0cm}\\
    \Large
    \textbf{Memoire de projet tuteuré}\vspace{1.5cm}\\
    \large
    \textbf{Fakhri \textsc{YAHIAOUI} - Roman \textsc{BOURSIER}}\vspace{1.5cm}\\
    Date de soutenance : le 09/06/2020\vspace{1.75cm}\\
  \end{center}\vspace{1.5cm}~\\
  \begin{tabular}{ll}
    \hspace{-0.45cm}Tuteur -- Université~:~&~Farès \textsc{BELHADJ}\\
  \end{tabular}
\end{titlepage}
\frontmatter
\chapter*{Résumé}
\markboth{\sc Résumé}{}
\addcontentsline{toc}{chapter}{Résumé} 

A faire en dernier ...


\chapter*{Remerciements}
\markboth{\sc Remerciements}{}
\addcontentsline{toc}{chapter}{Remerciements} 

TODO

%% Table des matières
\tableofcontents
%% La liste des figure est optionnelle (si votre rapport manque de
%% contenu ajouter ce type de pages sera perçu négativement)
\listoffigures
%% La liste des programmes est optionnelle (si votre rapport manque de
%% contenu ajouter ce type de pages sera perçu négativement)
\listofprograms
\mainmatter
\chapter*{Introduction}
\markboth{\sc Introduction}{}
\addcontentsline{toc}{chapter}{Introduction}

\chapter{L'estampe chinoise}


\chapter{Etat de l'art}
Chapeau


\section{Exemple de travaux}

Google DeepDream

Google quickdraw
https://quickdraw.withgoogle.com/

https://fr.wikipedia.org/wiki/DeepDream
https://github.com/google/deepdream


\section{Technologies existantes}

Neural Style Transfer
Gan


\chapter{Conclusion et Perspectives\label{chap-conclusion}}

\bibliographystyle{alpha}
\bibliography{memoire}
\end{document}
